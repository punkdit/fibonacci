% This line sets the project root file.
% 
% !TEX root = ../fib2.tex
% 
%------------------------------------------------------------------------------------------------------------%

%------------------------------------------------------------------------------------------------------------%
% Packages
%------------------------------------------------------------------------------------------------------------%

\usepackage{color}
\usepackage{amsmath,amsfonts,amssymb,amsthm}
\usepackage{graphicx}
\usepackage{fullpage}
%\usepackage{caption}
\PassOptionsToPackage{caption=false}{subfig}
\usepackage{subfig}
\usepackage{enumerate}

\usepackage{tikz}
\usetikzlibrary{arrows,decorations.pathmorphing,backgrounds,positioning,fit}
%\usepackage[square,comma,numbers,sort&compress]{natbib}
\usepackage{epstopdf} % to include .eps graphics files with pdfLaTeX
\usepackage{bm}  % Define \bm{} to use bold math fonts

\usepackage[pdfpagelabels,pdftex,bookmarks,breaklinks]{hyperref}
\definecolor{darkblue}{RGB}{0,0,127} % choose colors
\definecolor{darkgreen}{RGB}{0,150,0}
\hypersetup{colorlinks, linkcolor=darkblue, citecolor=darkgreen, filecolor=red, urlcolor=blue}
%\hypersetup{pdftitle=Title\ Goes\ Here}% add a title to the metadata

\usepackage{layout}

%------------------------------------------------------------------------------------------------------------%
% PGF Stuff
%------------------------------------------------------------------------------------------------------------%

%\pgfrealjobname{main}

%------------------------------------------------------------------------------------------------------------%
% Page Layout
%------------------------------------------------------------------------------------------------------------%

\addtolength{\textheight}{0\textheight}

%------------------------------------------------------------------------------------------------------------%
% Theorem Environments
%------------------------------------------------------------------------------------------------------------%

\newtheorem{theorem}{Theorem}
\newtheorem{proposition}[theorem]{Proposition}
\newtheorem{lemma}[theorem]{Lemma}
\newtheorem{corollary}[theorem]{Corollary}
\newtheorem{definition}[theorem]{Definition}
\newtheorem{remark}[theorem]{Remark}
\newtheorem{example}[theorem]{Example}

\newtheorem{claim}{Claim}[section]

\renewenvironment{proof}[1][Proof]{\noindent\textbf{#1.} }{\ $\Box$}

%------------------------------------------------------------------------------------------------------------%
% Macros
%------------------------------------------------------------------------------------------------------------%

% double-struck math font
\def\N{\mathbb{N}}
\def\Z{\mathbb{Z}}
\def\R{\mathbb{R}}
\def\C{\mathbb{C}}
\def\E{\mathbb{E}}

\def\e{\mathrm{e}}
\def\lg{\mathrm{lg}}

\newcommand{\Eref}[1]{Eq.~(\ref{#1})}
\newcommand{\Sref}[1]{Sec.~\ref{#1}}
\newcommand{\Fref}[1]{Fig.~\ref{#1}}
\newcommand{\Aref}[1]{Appendix~\ref{#1}}

\def\eps{\epsilon}
\def\th{^{\rm th}}
\def\st{^{\rm st}}
\def\rd{^{\rm rd}}
\def\cO{\mathcal{O}}
\DeclareMathOperator{\Tr}{Tr}
\DeclareMathOperator{\tr}{tr}
\DeclareMathOperator{\Prob}{Prob}

\newcommand{\ket}[1]{|{#1}\rangle}
\newcommand{\expect}[1]{\langle{#1}\rangle}
\newcommand{\bra}[1]{\langle{#1}|}
\newcommand{\ketbra}[2]{|{#1}\rangle\!\langle{#2}|}
\newcommand{\braket}[2]{\langle{#1}|{#2}\rangle}
\newcommand{\proj}[1]{\ketbra{#1}{#1}}

%------------------------------------------------------------------------------------------------------------%
% Comment fonts
%------------------------------------------------------------------------------------------------------------%

\newcommand{\cggb}[1]{\textcolor{blue}{#1}}
\newcommand{\sdb}[1]{\textcolor{red}{#1}}
\newcommand{\stf}[1]{\textcolor{green}{#1}}
